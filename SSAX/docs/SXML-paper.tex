% SXML->TEX created this file
\documentclass[10pt]{article}
\usepackage[T1]{fontenc}
\usepackage[latin1]{inputenc}
\usepackage{times}
\pagestyle{empty}
\makeatletter
\makeatother
\sloppy
\newcommand{\minitab}[2][l]{\begin{tabular}{#1}#2\end{tabular}}
% The standard article.cls leaves out too big margins
\setlength\topmargin{-10pt}
\setlength\textwidth{504pt}
\setlength\textheight{53\baselineskip}
\setlength\oddsidemargin{-0.3in}
\setlength\evensidemargin{-0.3in}

\begin{document}
\title{SXML Specification}
\author{Oleg Kiselyov\\FNMOC, Monterey, CA 93943\\oleg@pobox.com, oleg@acm.org}
\maketitle
\begin{abstract}
The present article specifies revision 3.0 of SXML. SXML is an abstract syntax tree of an XML document. SXML is also
a concrete representation of the XML Infoset in the form of
S-expressions. The generic tree structure of SXML lends itself to a
compact library of combinators for querying and transforming SXML.\\The master SXML specification file is written in SXML itself. The present paper is the result of translating that SXML code into \LaTeX, using an appropriate "stylesheet". A different stylesheet converted the specification to HTML. The master file, the stylesheets and the corresponding HTML and \LaTeX{} documents are available at  \texttt{http://pobox.com/\textasciitilde{}oleg/ftp/Scheme/xml.html} .\\Keywords: XML, XML parsing, XML Infoset, XML Namespaces, AST, SXML, Scheme.\end{abstract}
\section{Introduction}
An XML document is essentially a tree structure. The start and the end
tags of the root element enclose the document content,
which may include other elements or arbitrary character data.  Text
with familiar angular brackets is an external representation of an XML
document. Applications ought to deal with an internalized form:
an XML Information Set, or its specializations.  This form lets an
application locate specific data or transform an XML tree into another
tree, which can then be written out as an XML, HTML, PDF, etc.
document.

XML Information Set (Infoset) \cite{XML Infoset} is an
abstract datatype that describes information available in a
well-formed XML document.  Infoset is made of "information items",
which denote elements, attributes, character data, processing
instructions, and other components of the document. Each information
item has a number of associated properties, e.g., name, namespace
URI. Some properties -- for example, 'children' and 'attributes' --
are collections of other information items. Infoset describes only the
information in an XML document that is relevant to applications. The
default value of attributes declared in the DTD, parameter entities,
the order of attributes within a start-tag, and other data used merely
for parsing or validation are not included. Although technically
Infoset is specified for XML it largely applies to other
semi-structured data formats, in particular, HTML.

The hierarchy of containers comprised of text strings and other
containers greatly lends itself to be described by
S-expressions. S-expressions \cite{McCarthy} are easy to parse
into an internal representation suitable for traversal.  They have a
simple external notation (albeit with many a parentheses), which is
relatively easy to compose even by hand.  S-expressions have another
advantage: \emph{provided} an appropriate design, they can
represent Scheme code to be evaluated.  This code-data dualism is a
distinguished feature of Lisp and Scheme. 

SXML is a concrete instance of the XML Infoset. Infoset's goal is
to present in some form all relevant pieces of data and their \emph{abstract}, container-slot relationships with each other.  SXML
gives the nest of containers a concrete realization as S-expressions,
and provides means of accessing items and their properties. SXML is a
"relative" of XPath \cite{XPath} and DOM \cite{DOM},
whose data models are two other instances of the XML Infoset. SXML is
particularly suitable for Scheme-based XML/HTML authoring, SXPath
queries, and tree transformations. In John Hughes' terminology \cite{Hughes-PP}, SXML is a term implementation of evaluation of the XML
document.

\section{Notation}
We will use an Extended BNF Notation (EBNF) employed in the XML
Recommendation \cite{XML}. The following table summarizes the notation.

\begin{description}
\item [thing?] An optional \emph{thing}
\item [thing*] Zero or more \emph{thing}s
\item [thing+] One or more \emph{thing}s
\item [thing1 | thing2 | thing3] Choice of \emph{thing}s
\item [<thing>] A non-terminal of a grammar
\item [thing] A terminal of the grammar that is a Scheme identifier
\item ["thing"] A terminal of the grammar that is a Scheme string
\item [{\itshape thing}] A literal Scheme symbol
\item [\textbf{(} <A> <B>* <C>? \textbf{)}] An S-expression made of <A> followed by zero or more <B> and, afterwards, optionally by <C>
\item [\texttt{\textbf{\{}}{\itshape A} <B>*\texttt{ \textbf{\}}}] A tagged set: an S-expression made of {\itshape A} followed by zero or more instances of <B> in any order
\item [\textbf{(} <A>\textbf{ . }<B> \textbf{)}] An S-expression that is made by prepending <A> to an S-expression denoted by <B>
\item [\textbf{$MAKE{-}SYMBOL$(}<A>:<B>\textbf{)}] A symbol whose string representation consists of all
characters that spell <A> followed by the colon
character and by the characters that spell <B>. The \textbf{$MAKE{-}SYMBOL$(}\textbf{)} notation can be regarded a
meta-function that creates symbols.
\end{description}
\section{Grammar}
\begin{tabular}{rrcp{2.8in}}
{[}1{]} & \texttt{<TOP>} &  $::=$ & \texttt{\textbf{(} {\itshape *TOP*} <annotations>? <PI>* <comment>* <Element> \textbf{)} } \\
\end{tabular}
\\
This S-expression stands for the root of the SXML tree, a
document information item of the Infoset. Its only child element is
the root element of the XML document.

\begin{tabular}{rrcp{2.8in}}
{[}2{]} & \texttt{<Element>} &  $::=$ & \texttt{\textbf{(} <name> <annot-attributes>? <child-of-element>* \textbf{)} } \\
{[}3{]} & \texttt{<annot-attributes>} &  $::=$ & \texttt{\texttt{\textbf{\{}}{\itshape @} <attribute>* <annotations>?\texttt{ \textbf{\}}} } \\
{[}4{]} & \texttt{<attribute>} &  $::=$ & \texttt{\textbf{(} <name> "value"? <annotations>? \textbf{)} } \\
{[}5{]} & \texttt{<child-of-element>} &  $::=$ & \texttt{<Element> | "character data" | <PI> | <comment> | <entity> } \\
\end{tabular}
\\
These are the basic constructs of SXML.

\begin{tabular}{rrcp{2.8in}}
{[}6{]} & \texttt{<PI>} &  $::=$ & \texttt{\textbf{(} {\itshape *PI*} pi-target <annotations>? "processing instruction content string" \textbf{)} } \\
\end{tabular}
\\
The XML Recommendation specifies that processing instructions
 (PI) are distinct from elements and character data; processing
instructions must be passed through to applications. In SXML, PIs are
therefore represented by nodes of a dedicated type \texttt{*PI*}. DOM Level 2 treats processing instructions in a similar way.

\begin{tabular}{rrcp{2.8in}}
{[}7{]} & \texttt{<comment>} &  $::=$ & \texttt{\textbf{(} {\itshape *COMMENT*} "comment string" \textbf{)} } \\
{[}8{]} & \texttt{<entity>} &  $::=$ & \texttt{\textbf{(} {\itshape *ENTITY*} "public-id" "system-id" \textbf{)} } \\
\end{tabular}
\\
Comments are mentioned for completeness. A SSAX XML parser
\cite{SSAX}, among others, transparently skips the comments.
The XML Recommendation permits the parser to pass the comments to
an application or to completely disregard them. The present SXML grammar
admits comment nodes but does not mandate them by any means.\\ An <entity> node represents a reference to an
unexpanded external entity. This node corresponds to an unexpanded
entity reference information item, defined in Section 2.5 of \cite{XML Infoset}. Internal parsed entities are always expanded by the
XML processor at the point of their reference in the body of the
document.

\begin{tabular}{rrcp{2.8in}}
{[}9{]} & \texttt{<name>} &  $::=$ & \texttt{<LocalName> | <ExpName> } \\
{[}10{]} & \texttt{<LocalName>} &  $::=$ & \texttt{NCName } \\
{[}11{]} & \texttt{<ExpName>} &  $::=$ & \texttt{\textbf{$MAKE{-}SYMBOL$(}<namespace-id>:<LocalName>\textbf{)} } \\
{[}12{]} & \texttt{<namespace-id>} &  $::=$ & \texttt{\textbf{$MAKE{-}SYMBOL$(}"URI"\textbf{)} | user-ns-shortcut } \\
{[}13{]} & \texttt{<namespaces>} &  $::=$ & \texttt{\texttt{\textbf{\{}}{\itshape *NAMESPACES*} <namespace-assoc>*\texttt{ \textbf{\}}} } \\
{[}14{]} & \texttt{<namespace-assoc>} &  $::=$ & \texttt{\textbf{(} <namespace-id> "URI" {\itshape original-prefix}? \textbf{)} } \\
\end{tabular}
\\
An SXML <name> is a single symbol. It is
generally an expanded name \cite{XML-Namespaces}, which
conceptually consists of a local name and a namespace URI. The latter
part may be empty, in which case <name> is a NCName: a Scheme symbol whose spelling conforms to production [4]
of the XML Namespaces Recommendation \cite{XML-Namespaces}.
<ExpName> is also a Scheme symbol, whose string
representation contains an embedded colon that joins the local and the
namespace parts of the name. A \textbf{$MAKE{-}SYMBOL$(}"URI"\textbf{)} is a
Namespace URI string converted to a Scheme symbol. Universal Resource
Identifiers (URI) may contain characters (e.g., parentheses) that are
prohibited in Scheme identifiers. Such characters must be \%-quoted
during the conversion from a URI string to <namespace-id>. The original XML Namespace prefix of a QName \cite{XML-Namespaces} 
may be retained as an optional member {\itshape original-prefix} of a <namespace-assoc>
association. A user-ns-shortcut is a Scheme
symbol chosen by an application programmer to represent a namespace
URI in the application program. The SSAX parser lets the programmer
define (short and mnemonic) unique shortcuts for often long and unwieldy
Namespace URIs.

\section{Annotations}
\begin{tabular}{rrcp{2.8in}}
{[}15{]} & \texttt{<annotations>} &  $::=$ & \texttt{\texttt{\textbf{\{}}{\itshape @} <namespaces>? <annotation>*\texttt{ \textbf{\}}} } \\
{[}16{]} & \texttt{<annotation>} &  $::=$ & \texttt{~ } \emph{To be defined in the future}\\
\end{tabular}
\\
The XML Recommendation and related standards are not firmly
fixed, as the long list of errata and version 1.1 of XML
clearly show. Therefore, SXML has to be able to accommodate future
changes while guaranteeing backwards compatibility. SXML also ought to
permit applications to store various processing information (e.g.,
cached resolved IDREFs) in an SXML tree. A hash of ID-type attributes
would, for instance, let us implement efficient lookups in (SOAP-)
encoded arrays. To allow such extensibility, we introduce two new node
types: <annotations> and <annotation>. The
semantics of the latter is to be established in future versions of
SXML. Possible examples of an <annotation> are the unique
id of an element or the reference to element's parent.

The structure and the semantics of <annotations>
is similar to those of an attribute list. In a manner of speaking,
annotations are ``attributes'' of an attribute list. The tag
\texttt{@} marks a collection of ancillary data associated
with an SXML node. For an element SXML node, the ancillary collection
is that of attributes. A nested \texttt{@} list is therefore a
collection of ``second-level'' attributes -- annotations -- such as
namespace nodes, parent pointers, etc. This design seems to be in
accord with the spirit of the XML Recommendation, which uses XML
attributes for two distinct purposes. Genuine, semantic attributes
provide ancillary description of the corresponding XML element,
e.g.,

\begin{verbatim}
     <weight units='kg'>16</weight>
\end{verbatim}
On the other hand, attributes such as \texttt{xmlns}, \texttt{xml:prefix}, \texttt{xml:lang} and \texttt{xml:space} are
auxiliary, or being used by XML itself.  The XML Recommendation
distinguishes auxiliary attributes by their prefix
\texttt{xml}. SXML groups all such auxiliary attributes into a
\texttt{@}-tagged list inside the attribute list.

XML attributes are treated as a dust bin. For example, the XSLT
Recommendation allows extra attributes in \texttt{xslt:template}, provided these attributes are in a non-XSLT
namespace. A user may therefore annotate an XSLT template with his
own attributes, which will be silently disregarded by an XSLT
processor because the processor never looks for them. RELAX/NG
explicitly lets a schema author specify that an element may have more
attributes than given in the schema, provided those attributes come
from a particular namespace. The presence of these extra attributes
should not affect the XML processing applications that do not specifically
look for them. Annotations such as parent pointers and the source
location information are similarly targeted at specific
applications. The other applications should not be affected by the
presence or absence of annotations. Placing the collection of
annotations inside the attribute list accomplishes that goal.

Annotations can be assigned to an element and to an attribute
of an element. The following example illustrates the difference
between the two annotations, which, in the example, contain only one
annotation: a pointer to the parent of a node.

\begin{verbatim}
     (a (@ 
          (href "http://somewhere/" 
            (@ (*parent* a-node))       ; <annotations> of the attribute 'href'
            )
          (@ (*parent* a-parent-node))) ; <annotations> of the element 'a'
        "link")
\end{verbatim}
The <TOP> node may also contain annotations: for
example, <namespaces> for the entire document or an
index of ID-type attributes.

\begin{verbatim}
     (*TOP* (@ (id-collection id-hash)) (p (@ (id "id1")) "par1"))
\end{verbatim}
Annotations of the <TOP> element look exactly
like `attributes' of the element. That should not cause any confusion
because <TOP> cannot have genuine attributes. The SXML
element <TOP> is an abstract representation of the whole
document and does not correspond to any single XML element. Assigning
annotations, which look and feel like an attribute list, to the
<TOP> element does not contradict the Infoset
Recommendation, which specifically states that it is not intended to
be exhaustive. Attributes in general are not considered children of
their parents, therefore, even with our annotations the <TOP> element has only one child -- the root element.

\section{SXML Tree}
Infoset's information item is a sum of its properties. This makes
a list a particularly suitable data structure to represent an
item. The head of the list, a Scheme identifier, \emph{names} the
item. For many items this is their (expanded) name. For an information
item that denotes an XML element, the corresponding list starts with
element's expanded name, optionally followed by a collection of
attributes and annotations. The rest of the element item list
is an ordered sequence of element's children -- character data,
processing instructions, and other elements. Every child is unique;
items never share their children even if the latter have the identical
content.

A \texttt{parent} property of an Infoset information item
might seem troublesome. The Infoset Recommendation \cite{XML Infoset} specifies that element, attribute and other kinds of
information items have a property \texttt{parent}, whose value is
an information item that contains the given item in its \texttt{children} property. The property parent thus is an upward link
from a child to its parent. At first sight, S-expressions seem lacking
in that respect: S-expressions represent directed trees and trees
cannot have upward links. An article \cite{Parent-pointers}
discusses and compares five different methods of determining the
parent of an SXML node. The existence of these methods is a crucial
step in a constructive proof that SXML is a complete model of the XML
Information set and the SXML query language (SXPath) can fully
implement the XPath Recommendation.

Just as XPath does and the Infoset specification explicitly allows,
we group character information items into maximal text strings.  The
value of an attribute is normally a string; it may be omitted (in
case of HTML) for a boolean attribute, e.g., \texttt{<option checked>}.

We consider a collection of attributes an information item in its
own right, tagged with a special name \texttt{@}. The character '@'
may not occur in a valid XML name; therefore <annot-attributes> cannot be mistaken for a list that represents an element. An XML
document renders attributes, processing instructions, namespace
specifications and other meta-data differently from the element
markup. In contrast, SXML represents element content and meta-data
uniformly -- as tagged lists. RELAX-NG also aims to treat attributes
as uniformly as possible with elements. This uniform treatment, argues James
Clark \cite{RNG-Design}, is a significant factor
in simplifying the language.  SXML takes advantage of the fact that
every XML name is also a valid Scheme identifier, but not every
Scheme identifier is a valid XML name. This observation lets us
introduce administrative names such as \texttt{@}, \texttt{*PI*}, \texttt{*NAMESPACES*} without worrying about potential name
clashes. The observation also makes the relationship between XML and SXML
well-defined. An XML document converted to SXML can be reconstructed
into an equivalent (in terms of the Infoset) XML document. Moreover, due
to the implementation freedom given by the Infoset specification, SXML
itself is an instance of the Infoset.

Since an SXML document is essentially a tree structure, the SXML grammar of Section 3 can be presented in the following, more uniform form:

\begin{tabular}{rrcp{2.8in}}
{[}N{]} & \texttt{<Node>} &  $::=$ & \texttt{<Element> | <annot-attributes> | <attribute> | "character data: text string" | <namespaces> | <TOP> | <PI> | <comment> | <entity> | <annotations> | <annotation> } \\
\end{tabular}
\\
or as a set of two mutually-recursive datatypes, 
\texttt{Node} and \texttt{Nodelist}, where the latter is a
list of \texttt{Node}s: 

\begin{tabular}{rrcp{2.8in}}
{[}N1{]} & \texttt{<Node>} &  $::=$ & \texttt{\textbf{(} <name>\textbf{ . }<Nodelist> \textbf{)} | "text string" } \\
{[}N2{]} & \texttt{<Nodelist>} &  $::=$ & \texttt{\textbf{(} <Node> <Node>* \textbf{)} } \\
{[}N3{]} & \texttt{<name>} &  $::=$ & \texttt{<LocalName> | <ExpName> | {\itshape @} | {\itshape *TOP*} | {\itshape *PI*} | {\itshape *COMMENT*} | {\itshape *ENTITY*} | {\itshape *NAMESPACES*} } \\
\end{tabular}
\\
The uniformity of the SXML representation for elements,
attributes, and processing instructions simplifies queries and
transformations. In our formulation, attributes and processing
instructions look like regular elements with a distinguished
name. Therefore, query and transformation functions dedicated to
attributes become redundant.

A function \texttt{SSAX:XML->SXML} of a functional Scheme XML parsing
framework SSAX \cite{SSAX} can convert an XML document or a
well-formed part of it into the corresponding SXML form. The parser
supports namespaces, character and parsed entities, 
attribute value normalization, processing instructions and CDATA
sections.

\section{Namespaces}
The motivation for XML Namespaces is explained in an excellent article by James Clark \cite{Clark1999}. He says in part: 

\begin{quote}
The XML Namespaces Recommendation tries to improve this
situation by extending the data model to allow element type names and
attribute names to be qualified with a URI. Thus a document that
describes parts of cars can use \texttt{part} qualified by one URI; and a
document that describes parts of books can use \texttt{part} qualified by
another URI. I'll call the combination of a local name and a
qualifying URI a universal name. The role of the URI in a universal
name is purely to allow applications to recognize the name. There are
no guarantees about the resource identified by the URI. The XML
Namespaces Recommendation does not require element type names and
attribute names to be universal names; they are also allowed to be
local names.\\ ...\\ The XML Namespaces Recommendation expresses universal names in an
indirect way that is compatible with XML 1.0. In effect the XML
Namespaces Recommendation defines a mapping from an XML 1.0 tree where
element type names and attribute names are local names into a tree
where element type names and attribute names can be universal
names. The mapping is based on the idea of a prefix. If an element
type name or attribute name contains a colon, then the mapping treats
the part of the name before the colon as a prefix, and the part of the
name after the colon as the local name. A prefix \texttt{foo}
refers to the URI specified in the value of the \texttt{xmlns:foo}
attribute. So, for example\begin{verbatim}
     <cars:part xmlns:cars='http://www.cars.com/xml'/>
\end{verbatim}
maps to\begin{verbatim}
     <{http://www.cars.com/xml}part/>
\end{verbatim}
Note that the \texttt{xmlns:cars} attribute has been removed by the mapping.\end{quote}
Using James Clark's terminology, SXML as defined by [N1] is
precisely that tree where element type names and attribute names can
be universal names.  According to productions [N3] and [9-12], a
universal name, <name>, is either a local name or an expanded
name. Both kinds of names are Scheme identifiers. A local name has no
colon characters in its spelling. An expanded name is spelled with at
least one colon, which may make the identifier look rather odd. In
SXML, James Clark's example will appear as follows:\begin{verbatim}
     (http://www.cars.com/xml:part)
\end{verbatim}
or, somewhat redundantly, \begin{verbatim}
     (http://www.cars.com/xml:part
       (@ (@ (*NAMESPACES* (http://www.cars.com/xml "http://www.cars.com/xml" cars)))))
\end{verbatim}


Such a representation also agrees with the Namespaces Recommendation \cite{XML-Namespaces}, which says: "Note that the prefix
functions only as a placeholder for a namespace name. Applications
should use the namespace name, not the prefix, in constructing names
whose scope extends beyond the containing document."

It may be unwieldy to deal with identifiers such as \texttt{http://www.cars.com/xml:part}. Therefore, an application that
invokes the SSAX parser may tell the parser to map the URI \texttt{http://www.cars.com/xml} to an application-specific namespace shortcut user-ns-shortcut, e.g., \texttt{c}. The parser will then produce\begin{verbatim}
     (c:part (@ (@ (*NAMESPACES* (c "http://www.cars.com/xml")))))
\end{verbatim}
To be more precise, the parser will return just\begin{verbatim}
     (c:part)
\end{verbatim}
If an application told the parser how to map \texttt{http://www.cars.com/xml}, the application can keep this mapping in
its mind and will not need additional reminders.On the other hand, we should not be afraid of SXML node
names such as \texttt{http://www.cars.com/xml:part}. These names
are Scheme symbols. No matter how long the name of a symbol may be, it
is fully spelled only once, in the symbol table. All other occurrences
of the symbol are short references to the corresponding slot in
the symbol table.



We must note a 1-to-1 correspondence between user-ns-shortcuts and the corresponding namespace URIs. This is
generally not true for XML namespace prefixes and namespace URIs. A
user-ns-shortcut uniquely represents the
corresponding namespace URI within the document, but an XML namespace
prefix does not. For example, different XML prefixes may specify the
same namespace URI; XML namespace prefixes may be redefined in
children elements. The other difference between user-ns-shortcuts and XML namespace prefixes is that the latter
are at the whims of the author of the document whereas the namespace
shortcuts are defined by an XML processing application. The shortcuts
are syntactic sugar for namespace URIs.

The list of associations between namespace IDs and namespace
URIs (and, optionally, original XML Namespace prefixes) is an optional
member of an <annotations>. For regular elements,
<namespaces> will contain only those namespace
declarations that are relevant for that element. Most of the time
however <namespaces> in the <annotations> will be absent.

The node <namespaces>, if present as an annotation
of the <TOP> element, must contain
<namespace-assoc> for the whole document. It may
happen that one namespace URI is associated in the source XML document
with several namespace prefixes. There will be then several
corresponding <namespace-assoc> differing only in
the {\itshape original-prefix} part.

The topic of namespaces in (S)XML and (S)XPath is thoroughly
discussed in \cite{Lisovsky-NS} and \cite{SXML-NS}.

\section{Case-sensitivity of SXML names}
XML is a case-sensitive language. The names of XML elements,
attributes, targets of processing instructions, etc. may be
distinguished solely by their case. These names however are
represented as Scheme identifiers in SXML. Although Scheme is
traditionally a case-insensitive language, the use of Scheme symbols
for XML names poses no contradictions. According to R5RS \cite{R5RS}, \begin{quote}
(symbol->string symbol) returns the name of symbol as a string. If
the symbol was part of an object returned as the value of a literal
expression (section 4.1.2) or by a call to the read procedure, and its
name contains alphabetic characters, then the string returned will
contain characters in the implementation's preferred standard
case -- some implementations will prefer upper case, others lower
case. If the symbol was returned by string->symbol, the case of
characters in the string returned will be the same as the case in the
string that was passed to string->symbol.\end{quote}


Thus, R5RS explicitly permits case-sensitive symbols: \texttt{(string->symbol "a")} is always different from \texttt{(string->symbol "A")}. SXML uses such case-sensitive symbols for
all XML names. SXML-compliant XML parsers must preserve the case of
all names when converting them into symbols. A parser may use the R5RS
procedure \texttt{string->symbol} or other available means.

Reading and writing SXML documents may require special care. The cited R5RS section allows a Scheme system to alter the case of symbols found in a literal expression or read from input ports. Entering SXML names on case-insensitive systems requires the use of a bar notation, \texttt{string->symbol} or
other standard or non-standard ways of producing case-sensitive
symbols. A SSAX built-in test suite is a highly portable example of entering literal case-sensitive SXML names. Such workarounds, however simple, become unnecessary on Scheme systems that support DSSSL. According to the DSSSL standard,\begin{quote}
7.3.1. Case Sensitivity. Upper- and lower-case forms of a
letter are always distinguished. NOTE 5: Traditionally Lisp systems
are case-insensitive.\end{quote}
In addition, a great number of Scheme systems offer a
case-sensitive reader, which often has to be activated through a
compiler option or pragma. A web page \cite{Scheme-case-sensitivity} discusses case sensitivity of various Scheme systems in detail.

\section{Normalized SXML}
Normalized SXML is a proper subset of SXML optimized for
efficient processing. An SXML document in a normalized form must
satisfy a number of additional restrictions. Most of the restrictions
concern the order and the appearance of the <annot-attributes> and <annotations> within an
<Element> node, productions [2-4]. In the spirit of the
relational database model, we introduce a number of increasingly
rigorous normal forms for SXML expressions. An SXML document in normal
form N is also in normal form M for any M<N. The higher normal forms
impose more constraints on the structure of SXML expressions but in
return permit faster access.

The most permissive 0NF may be considered a relaxation of the
SXML grammar. The form does not mandate the relative order of <annot-attributes>. The attribute list, if present, may be
inter-mixed with <child-of-element>. SGML provides two
equal forms for boolean attributes: minimized, e.g., \texttt{<OPTION
checked>} and full, \texttt{<OPTION checked="checked">}. XML
mandates the full form only, whereas HTML allows both, favoring the
former. 0NF SXML supports the minimized form along with the full one:
\texttt{(OPTION (@ (checked)))} and \texttt{(OPTION (@ (checked
"checked")))}.

In the 1NF, optional <annot-attributes> if
present must precede all other components of an <Element> SXML node. This is the order reflected in Production [2].  Boolean
attributes must appear in their full form, e.g., \texttt{(OPTION (@
 (checked "checked")))}. The 1NF is the "recommended" normal
form.

The 2NF makes <annot-attributes> a required
component of an SXML element node. If an element has no attributes nor annotations, <annot-attributes> shall be specified as \texttt{(@)}. In the 2NF, <comment> and <entity> nodes must be absent. Parsed entities should be expanded, even if
they are external.

In the third normal form 3NF all text strings must be
joined into maximal text strings: no <Nodelist> shall
contain two adjacent text-string nodes.

The normal forms make it possible to access SXML items in
efficient ways. If an SXML document is known to be in the 3NF, for
example, an application never has to check for the existence of
 <annot-attributes>. Checking for child nodes and
retrieving text data are simplified as well.

\section{Examples}
Simple examples:\begin{verbatim}
     (some-name)        ; An empty element without attributes (in 0NF and 1NF)
     (some-name (@))    ; The same but in the normalized (2NF) form
\end{verbatim}


The figure below shows progressively more complex examples.

\begin{table}[ht]
\begin{tabular}{@{\extracolsep{-25pt}}|l|l|}
\hline
\multicolumn{1}{c}{\minitab[c]{XML}} & \multicolumn{1}{c}{\minitab[c]{SXML}}\\\hline
\multicolumn{1}{l}{\minitab[l]{\texttt{\small <WEIGHT~unit="pound">}\\\texttt{\small ~~<NET~certified="certified">67</NET>}\\\texttt{\small ~~<GROSS>95</GROSS>}\\\texttt{\small </WEIGHT>}\\}} & \multicolumn{1}{l}{\minitab[l]{\texttt{\small (WEIGHT~(@~(unit~"pound"))}\\\texttt{\small ~~(NET~(@~(certified))~67)}\\\texttt{\small ~~(GROSS~95)}\\\texttt{\small )}\\}}\\\hline
\multicolumn{1}{l}{\minitab[l]{\texttt{\small ~<BR/>}\\}} & \multicolumn{1}{l}{\minitab[l]{\texttt{\small (BR)}\\}}\\\hline
\multicolumn{1}{l}{\minitab[l]{\texttt{\small ~<BR></BR>}\\}} & \multicolumn{1}{l}{\minitab[l]{\texttt{\small (BR)}\\}}\\\hline
\multicolumn{1}{l}{\minitab[l]{\texttt{\small ~<P>}\\\texttt{\small <![CDATA[<BR>$\backslash$r$\backslash$n}\\\texttt{\small <![CDATA[<BR>]]]]>\&gt;~</P>}\\}} & \multicolumn{1}{l}{\minitab[l]{\texttt{\small (*TOP*}\\\texttt{\small ~~(P~"<BR>$\backslash$n<![CDATA[<BR>]]>~"))}\\}}\\\hline
\multicolumn{2}{c}{\minitab[c]{\\ ~An example from the XML Namespaces Recommendation}}\\\hline
\multicolumn{1}{l}{\minitab[l]{\texttt{\small <!--~initially,~the~default}\\\texttt{\small ~~~~~namespace~is~'books'~-->}\\\texttt{\small <book~xmlns='urn:loc.gov:books'}\\\texttt{\small ~~xmlns:isbn='urn:ISBN:0-395-36341-6'>}\\\texttt{\small ~~<title>Cheaper~by~the~Dozen</title>}\\\texttt{\small ~~<isbn:number>1568491379</isbn:number>}\\\texttt{\small ~~<notes>}\\\texttt{\small ~~<!--~make~HTML~the~default~namespace~}\\\texttt{\small ~~~~~~~for~some~commentary~-->}\\\texttt{\small ~~~~~<p~xmlns='urn:w3-org-ns:HTML'>}\\\texttt{\small ~~~~~~~This~is~a~<i>funny</i>~book!}\\\texttt{\small ~~~~~</p>}\\\texttt{\small ~~</notes>}\\\texttt{\small </book>}\\}} & \multicolumn{1}{l}{\minitab[l]{\texttt{\small (*TOP*}\\\texttt{\small ~~(urn:loc.gov:books:book}\\\texttt{\small ~~~~(urn:loc.gov:books:title}\\\texttt{\small ~~~~~~~~"Cheaper~by~the~Dozen")}\\\texttt{\small ~~~~(urn:ISBN:0-395-36341-6:number~"1568491379")}\\\texttt{\small ~~~~(urn:loc.gov:books:notes}\\\texttt{\small ~~~~~~(urn:w3-org-ns:HTML:p}\\\texttt{\small ~~~~~~~~~"This~is~a~"}\\\texttt{\small ~~~~~~~~~~(urn:w3-org-ns:HTML:i~"funny")}\\\texttt{\small ~~~~~~~~~"~book!"))))}\\}}\\\hline
\multicolumn{2}{c}{\minitab[c]{\\ ~Another example from the XML Namespaces Recommendation}}\\\hline
\multicolumn{1}{l}{\minitab[l]{\texttt{\small <RESERVATION~}\\\texttt{\small ~~xmlns:HTML=}\\\texttt{\small ~~~'http://www.w3.org/TR/REC-html40'>}\\\texttt{\small <NAME~HTML:CLASS="largeSansSerif">}\\\texttt{\small ~~~~Layman,~A</NAME>}\\\texttt{\small <SEAT~CLASS='Y'~}\\\texttt{\small ~~HTML:CLASS="largeMonotype">33B</SEAT>}\\\texttt{\small <HTML:A~HREF='/cgi-bin/ResStatus'>}\\\texttt{\small ~~~~Check~Status</HTML:A>}\\\texttt{\small <DEPARTURE>1997-05-24T07:55:00+1}\\\texttt{\small </DEPARTURE></RESERVATION>}\\}} & \multicolumn{1}{l}{\minitab[l]{\texttt{\small (*TOP*}\\\texttt{\small ~~(@~(*NAMESPACES*~}\\\texttt{\small ~~~~~~~(HTML~"http://www.w3.org/TR/REC-html40")))}\\\texttt{\small ~~(RESERVATION}\\\texttt{\small ~~~~(NAME~(@~(HTML:CLASS~"largeSansSerif"))}\\\texttt{\small ~~~~~~"Layman,~A")}\\\texttt{\small ~~~~(SEAT~(@~(HTML:CLASS~"largeMonotype")}\\\texttt{\small ~~~~~~~~~~~~~(CLASS~"Y"))}\\\texttt{\small ~~~~~~~"33B")}\\\texttt{\small ~~~~(HTML:A~(@~(HREF~"/cgi-bin/ResStatus"))}\\\texttt{\small ~~~~~~~"Check~Status")}\\\texttt{\small ~~~~(DEPARTURE~"1997-05-24T07:55:00+1")))}\\}}\\\hline
\end{tabular}
\end{table}

\subsection*{Acknowledgments}
I am indebted to Kirill Lisovsky for insightful
discussions and suggestions. This work has been supported in part by
the National Research Council Research Associateship Program and Naval
Postgraduate School.

\begin{thebibliography}{}
\bibitem{McCarthy} John McCarthy. Recursive Functions of Symbolic Expressions
and Their Computation by Machine, Part I. Comm. ACM, 3(4):184-195, April 1960. \texttt{http://www-formal.stanford.edu/jmc/recursive/recursive.html} 
\bibitem{Clark1999} Jim Clark. XML Namespaces. February 4, 1999. \texttt{http://www.jclark.com/xml/xmlns.htm} 
\bibitem{RNG-Design} James Clark, The Design of RELAX NG. December 6, 2001. \texttt{http://www.thaiopensource.com/relaxng/design.html} 
\bibitem{Hughes-PP} John Hughes, The Design of a Pretty-printing Library. Advanced Functional Programming, J. Jeuring and E. Meijer, Eds. Springer Verlag, LNCS 925, 1995, pp. 53-96. \texttt{http://www.cs.chalmers.se/\textasciitilde{}rjmh/Papers/pretty.html} 
\bibitem{R5RS} R. Kelsey, W. Clinger, J. Rees (eds.), Revised5 Report on
                      the Algorithmic Language Scheme. Higher-Order and
                      Symbolic Computation, Vol. 11, No. 1, September, 1998
                      and
                      ACM SIGPLAN Notices, Vol. 33, No. 9, October, 1998. \texttt{http://www.schemers.org/Documents/Standards/R5RS/} 
\bibitem{SSAX} Oleg Kiselyov. Functional XML parsing framework: SAX/DOM and
SXML parsers with support for XML Namespaces and validation. September
5, 2001. \texttt{http://pobox.com/\textasciitilde{}oleg/ftp/Scheme/xml.html\#XML-parser} 
\bibitem{SXML-short-paper} Oleg Kiselyov. XML and Scheme. An introduction to SXML and SXPath;
illustration of SXPath expressiveness and comparison with
XPath. September 17, 2000. \texttt{http://pobox.com/\textasciitilde{}oleg/ftp/Scheme/SXML-short-paper.html} 
\bibitem{Parent-pointers} Oleg Kiselyov. On parent pointers in SXML trees. February 12, 2003. \texttt{http://pobox.com/\textasciitilde{}oleg/ftp/Scheme/xml.html\#parent-ptr} 
\bibitem{Scheme-case-sensitivity} Kirill Lisovsky. Case sensitivity of Scheme systems. \texttt{http://pair.com/lisovsky/scheme/casesens/index.html} 
\bibitem{Lisovsky-NS} Kirill Lisovsky. Namespaces in XML and SXML.  \texttt{http://pair.com/lisovsky/xml/ns/} 
\bibitem{SXML-NS} Namespaces in SXML and (S)XPath. Discussion thread on the SSAX-SXML mailing list. May 28, 2002 and June 7, 2002. \texttt{http://sourceforge.net/mailarchive/forum.php?thread\_id=759249\&forum\_id=599}  \texttt{http://sourceforge.net/mailarchive/forum.php?thread\_id=789156\&forum\_id=599} 
\bibitem{Annotations} An alternative syntax for aux-list. Discussion thread on the SSAX-SXML mailing list. January 5-14, 2004. \texttt{http://article.gmane.org/gmane.lisp.scheme.ssax-sxml/37} 
\bibitem{DOM} World Wide Web Consortium. Document Object Model (DOM) Level 1
Specification. W3C Recommendation. \texttt{http://www.w3.org/TR/REC-DOM-Level-1} 
\bibitem{XML} World Wide Web Consortium. Extensible Markup Language (XML)
1.0 (Second Edition). W3C Recommendation. October 6, 2000. \texttt{http://www.w3.org/TR/REC-xml} 
\bibitem{XML Infoset} World Wide Web Consortium. XML Information Set.  W3C Recommendation. 24 October 2001. \texttt{http://www.w3.org/TR/xml-infoset} 
\bibitem{XML-Namespaces} World Wide Web Consortium. Namespaces in XML. W3C Recommendation. January 14, 1999. \texttt{http://www.w3.org/TR/REC-xml-names/} 
\bibitem{XPath} World Wide Web Consortium. XML Path Language (XPath).
Version 1.0. W3C Recommendation. November 16, 1999. \texttt{http://www.w3.org/TR/xpath} 
\end{thebibliography}
\section{Changes from the previous version}
Following the suggestion by Matthias Radestock, <aux-list> is renamed into <annotations>, and <aux-node> into <annotation>. What used to be called
<attribute-list> is now <annot-attributes>.

Jim Bender suggested annotations in <PI> nodes, for
example, to record the location of the processing instruction in the
document source.

We have added a detailed discussion of annotations on a single
attribute, and of <TOP> annotations.

We have introduced the notation for a tagged set \texttt{\textbf{\{}}{\itshape a} <B> <C>\texttt{ \textbf{\}}} to precisely specify the syntax of <annot-attributes> and <annotations>.

Previously SXML defined an attributes list and an aux-list as:

\begin{verbatim}
     [3]  <attributes-list> ::= ( @ <attribute>* )
     [15] <aux-list> ::= ( @@ <namespaces>? <aux-node>* )
\end{verbatim}
Both lists looked alike, as tagged associative lists. Attribute
lists were tagged with a distinguished symbol \texttt{@} and
aux-lists were tagged with a distinguished symbol \texttt{@@}. Both lists were ``improper'' children of their parent SXML
element. Both lists were optional.  An aux-list contained `auxiliary'
associations, e.g., the information about original namespace prefixes
or the pointer to the parent SXML element. Here is an example of an SXML
element with both attributes-list and aux-list:

\begin{verbatim}
     (tag (@ (attr "val")) (@@ (*parent* val)) kid1 kid2)
\end{verbatim}
In a normalized SXML (2NF), both lists had to be present, and
appear in the right order among the children of an element. The empty
attributes-list had to be coded as \texttt{(@)} and the empty
aux-list had to be coded as \texttt{(@@)}.

In the present version, <attributes-list> is renamed into
<annot-attributes>, and <aux-list> into
<annotations>.  Both <annot-attributes> and
<annotations> are tagged with the same symbol: \texttt{@}. Annotations may no longer appear among the children of an
element. Rather, annotations may only appear inside
<annot-attributes> and at the top level. The previous
example reads now as follows:

\begin{verbatim}
     (tag (@ (attr "val") (@ (*parent* val))) kid1 kid2)
\end{verbatim}
The new format for annotations makes it easier to skip them
when they are not needed. If an SXML application only looks up attributes by
their names, the change in syntax is transparent.  The transparency of
annotation nodes in SXPath is one reason for using the same symbol
\texttt{@} to tag both <annot-attributes> and <annotations>. In more detail,
this point is discussed in \cite{Annotations}. It seems that
most of the existing SXML processing code will not be affected by the
change. Placing annotations inside <annot-attributes>
seems to make it easier to process them during SXSLT traversals. In
fact, that was the original motivation for the change in syntax.

The new format makes SXML more space efficient, for documents
where most elements have no annotations nor attributes. Indeed, an
SXML node without attributes and annotations previously had the
following 3NF form:

\begin{verbatim}
     (tag (@) (@@) data)
\end{verbatim}
Now, the same node is realized as \texttt{(tag (@) data)}A detailed discussion is given in \cite{Annotations}. 

Previously, annotations were mandatory in 3NF. If a node had
no annotations, its aux-list had to be specified as \texttt{(@@)}. The aux-list had to appear at the fixed position, as
the third member of an SXML element node, so that we could easily
locate the proper children of a node without extra tests. In the
present version of SXML, <annotations> are included in
<annot-attributes>. Attribute lists are considered
unordered and are typically handled with the help of \texttt{assq}. Specifying a fixed position for annotations or making them
mandatory in the 3NF no longer seems reasonable nor would it
noticeably speed up SXML processing. Thus, if an element has no
annotations, <annotations> is absent. If the
element also has no attributes, the empty \texttt{annot-attributes} list \texttt{(@)} must still be present, in 2NF
and higher.

\end{document}
